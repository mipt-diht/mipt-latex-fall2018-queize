\documentclass[11pt, a5paper]{article}
\usepackage[utf8]{inputenc}
\usepackage{mathrsfs}
\usepackage{amsmath}
\usepackage{amsthm}
\usepackage{gensymb}
\usepackage{amssymb}
\DeclareMathOperator{\mat}{Mat}
\usepackage[russian]{babel}
\usepackage[
left	= 1cm,
right	= 1cm,
top	= 1cm,
bottom	= 2cm
]{geometry}
\newcounter{abc}
\setcounter{abc}{2}
\newtheorem{theorem}{Теорема}[section]
\theoremstyle{definition}\newtheorem{definition}{Определение}
\newcounter{Number}[section]
\newenvironment{tasks}[1]{\stepcounter{Number} \underline{\textbf{Задача №\arabic{Number}. #1 }} \\[8pt]}{\\[8pt]}


\begin{document}

\section{Центральная предельная теорема}
	\begin{theorem}[Линдеберга]
	 Пусть $\{\xi_k\}_{k \geqslant 1}$~--- независимые случайные величины, $\mathsf{E} \xi_k^2 <+ \infty,~\forall k$, обозначим $m_k=\mathsf{E} \xi_k,~\delta_k^2=\mathsf{D} \xi_k;~S_n=\sum\limits_{i=1}^{n}S_i;~\mathsf{D}_n^2 = \sum\limits_{k=0}^{n}\delta_k^2$ и $F(x)$~--- функция распределения $\xi_k$. Пусть выполнено условие Линдербега, т.е.
	 $$\forall\xi>0,\frac{1}{\mathsf{D}^2}\sum\limits_{k=1}^{n}~ \int\limits_{\{x:\mid x-m_k\mid >\varepsilon \mathsf{D}_n\}} (x-m_k)^2\,dF(x) \xrightarrow[n \rightarrow \infty]{} 0 $$
	 Тогда  $$ \frac{S_n-\mathsf{E}S_n}{\sqrt{\mathsf{D}S_n}}\overset{\text{d}}{\longrightarrow} \mathcal{N} (0,1),~n \rightarrow \infty.$$
	\end{theorem}
	\section{Гауссовский случайный вектор}
	\begin{definition}
Слаучайный вектор $\vec{\xi}\sim \mathcal{N}(m,\sum)$~--- гауссовский, если его характеристическая функция $\varphi_\xi(\vec{t})=e^{i(\vec{m},\vec{t}) -\frac{1}{2}(\sum\vec{t},\vec{t})},~\vec{m} \in \mathcal{R}^n,~\sum$~--- симметрическая, неотрицательно-определенная матрица.
	\end{definition}
	\begin{definition}
Случайный вектор $\vec{\xi}$~--- гауссовский, если он представляется в следующем виде: $\vec{\xi} = A\vec{\eta} + \vec{B}$, где $\vec{B} \in \mathbb{R}^n,A \in \mat (n\times m)$ и $\vec{\eta} = \{\vec{\eta_1}, \ldots,\vec{\eta_m} \}$~--- независимые, $n \sim \mathcal{N}(0,1)$
	\end{definition}
	\begin{definition}
Случайный вектор $\vec{\xi}$~--- гауссовский, если $\forall \lambda \in \mathbb{R}^n$ случайный вектор $(\vec{\lambda}, \vec{\xi})$ имеет нормальное распределение
	\end{definition}
	\begin{theorem}[об эквивалентности определений гауссовского вектора]
		Предыдущие три определния эквиваленты
	\end{theorem}
	\section{Задачи по астрономии}
	\begin{tasks}{Загадочный круг}
	Установите астрономический азимут восхода звезды $\varepsilon$ CMa $(6^h58^m38^s,$ $-28\degree58')$ при наблюдении  из самой северной равноудаленной от Санкт-Петербурга $(59\degree57'~\text{с.ш.},~ 30\degree19'$ $\text{в.д.})$ и Красной поляны $(43\degree41'~\text{с.ш.} $ , $40\degree11'~\text{в.д.})$ точки земной поверхности. Атмосферой пренебрегите, а Земля --- шар.
	\end{tasks}
	\begin{tasks}{К Сатурну!} 
	Космический корабль запустили с поверхности Земли к Сатурну по наиболее энергетически выгодной траектории. При движении по орбите корабль пролетел мимо астероида-троянца (624) Гектор. \\[8pt]
	Определите большую полуось и экцентреситет полученной орбиты, скорость старта с поверхности, а также угол между направлением Солнце и на Сатурн в момент старта корабля. Орбиты планет считать круговыми. Оцените относительную скорость корбля и астероида в момент сближения
	\end{tasks}
	\begin{tasks}{H \Roman{abc}}
	Предположим, что за пределами солнечного круга кривая вращения Галактики плоская, параметр плато $v$ = 240 км/с. Пусть известно, что диск нейтроального водорода простирается до галакто-центрического расстояния $R_\text{max}$ = 50 кпк. Мы наблюдаем облако нейтрального водорода на галактической долготе $l$ = 140\degree. Оцените минимально возможное значение лучевой скорости этого облака. 
	\end{tasks}
	\begin{tasks}{Обратный комптон-эффект}
	Обратным комптон-эффектом (ОЭК) называют явление рассеяния фотона на ултрарелятивистском свободном электроне, при котором происходит перенос энергии от электрона к фотону. Рассмотрим ОЭК для фотонов реликтового излучения. При какой энергии электронов в направленном пучке рассеяное излучение можно будет зарегистрировать на фотоприёмнике?
	\end{tasks}
	\section{Отзыв}
	\begin{itemize}
		\item \textbf{Наверное, один из самых полезных курсов на физтехе, который с большой вероятностью пригодится в жизни
		\item Хотелось бы побольше примеров. Теория всегда хорошо, но практика находится  в большем приоритете.
		\item Самый лучший лектор --- это твой сверстник: с ним всегда можно найти общий язык}
	\end{itemize}
\end{document}


