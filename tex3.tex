\documentclass[11pt,a4paper]{article}
\usepackage[utf8]{inputenc}
\usepackage[russian]{babel}
\usepackage{gensymb}
\usepackage{wrapfig}
\usepackage{amsmath}
\usepackage{amssymb}
\usepackage{tabularx}
\usepackage{mathtools}
\usepackage{graphicx}
\usepackage{physics}
\usepackage{subcaption}
\usepackage[left=2cm,right=2cm,top=2cm,bottom=2cm]{geometry}
\usepackage[
	labelfont = bf,
	labelsep = period,	
	justification = centering,
	textfont = sl]{caption}
	
\newcolumntype{C}[0]{>{\centering\arraybackslash}X}
\renewcommand\tabularxcolumn[1]{m{#1}}

\begin{document}
	\section{Обзор планеты гиганта}
	\begin{wrapfigure}[]{l}{0.4\textwidth}
		\vspace{-1pc}
		\includegraphics[width = 0.4\textwidth]{img/saturn}
		\caption{Снимок планеты со станции <<Кассини>>}
		\label{pic:Saturn}
	\end{wrapfigure}	
	 \textbf{Сатурн} --- шестая планета от Солнца и вторая по размерам планета в Солнечной системе после Юпитера. Сатурн, а также Юпитер, Уран и Нептун, классифицируются \textit{как газовые гиганты}. Сатурн назван в честь римского бога земледелия. На Рис.\,\ref{pic:Saturn} можно увидеть как выглядит фотография газового гиганта, сделанного при помощи автоматической станции <<Касини>> \par
	 В основном Сатурн состоит из водорода $\mathrm{H_2}$, с примесями гелия $\mathrm{He}$ и следами воды, метана $\mathrm{CH_4}$, аммиака $\mathrm{NH_3}$ и тяжёлых элементов. Внутренняя область представляет собой относительно небольшое ядро из железа $\mathrm{Fe}$, никеля $\mathrm{Ni}$ и льда, покрытое тонким слоем металлического водорода и газообразным внешним слоем. \par
	 Сатурн обладает заметной системой колец, состоящей главным образом из частичек льда, меньшего количества тяжёлых элементов и пыли. Вокруг планеты обращается 62 известных на данный момент спутника. \textbf{Титан} --- самый крупный из них, а также второй по размерам спутник в Солнечной системе (после спутника Юпитера, Ганимеда), который превосходит по своим размерам Меркурий и обладает единственной среди спутников планет Солнечной системы плотной атмосферой.\par
	 \begin{wrapfigure}[]{r}{0.4\textwidth}
	 	\vspace{-1pc}
	 	\includegraphics[width = 0.4\textwidth]{img/img18}
	 	\caption{Внутреннее строение Сатурна}	
	 \end{wrapfigure}
	 В глубине атмосферы Сатурна растут давление и температура, а водород переходит в жидкое состояние, однако этот переход является постепенным. На глубине около 30 тыс. км водород становится металлическим (давление там достигает около 3 миллионов атмосфер). В центре планеты находится массивное ядро из твердых и тяжёлых материалов --- силикатов, металлов и, предположительно, льда. Его масса составляет приблизительно от 9 до 22 масс Земли. Температура ядра достигает 11 700\,\degree C, а энергия, которую Сатурн излучает в космос, в 2.5 раза больше энергии, которую планета получает от Солнца.
	 \section{Спутники} 
	По состоянию на февраль 2010 г. известно 62 спутника Сатурна. Крупнейшие спутники --- Мимас, Энцелад, Тефия, Диона, Рея, Титан и Япет --- были открыты к 1789 году, однако и по сегодняшний день остаются основными объектами исследований. Распределение по массам соответствует распределению по диаметрам. Наибольшим эксцентриситетом орбиты обладает Титан, наименьшим --- Диона и Тефия. Все спутники c известными параметрами находятся выше синхронной орбиты, что приводит к их постепенному удалению.\par
	Самый крупный из спутников --- \textit{Титан}. Также он является вторым по величине в Солнечной системе в целом, после спутника Юпитера Ганимеда. 
	\begin{figure}[h!]
		\begin{tabularx}{\textwidth}{|C|C|C|C|C|}
			\hline
			Название спутника & Большая полуось $a$, тыс.км & Эксцентриситет, $e$ & Период обращения $T$, земных суток & Наклон орбиты $i$, \degree \\
			\hline
			Мимас & 185.539 & 0.0196 & 0.942 & 1.574 \\
			\hline
			Энцелад & 237.948 & 0.0047 & 1.370 & 0.019 \\
			\hline
			Тефия & 294.672 & 0.0001 & 1.887802 & 1.12 \\
			\hline
			Диона & 377.4  & 0.0022 & 2.77 & 0.019 \\
			\hline
		\end{tabularx}
	\end{figure}
	\section{Интегрирование}
	Пусть на отрезке $[-10,10]$ задана уравнением (\ref{eq:equation1})  кусочно-непрерывная функция $f(x)$:
	\begin{equation}
	f(x) = 
	\begin{cases}
	x, & 10 \leqslant x \leqslant 6; \\
	x^2, &  6 \leqslant x \leqslant 2; \\
	x^3, & 2 \leqslant x \leqslant 2; \\
	x^4, & 2 \leqslant x \leqslant 6; \\
	x^5, & 6 \leqslant x \leqslant 10;
	\end{cases} 
	\label{eq:equation1}
	\end{equation}
	Производная будет также иметь вид кучно-непрерывной функции. Для того, чтобы получить вид производной необходимо продифференцировать функцию на каждом из промежутке, на котором она задана. Сама же производная будет задаваться уравнение (\ref{eq:equation2}):
	\begin{equation}
	f'(x) = 
	\begin{cases}
	1, & -10 \leqslant x \leqslant -6; \\
	2x, &  -6 \leqslant x \leqslant -2; \\
	3x^2, & -2 \leqslant x \leqslant 2; \\
	4x^3, & 2 \leqslant x \leqslant 6; \\
	5x^4, & 6 \leqslant x \leqslant 10;
	\end{cases}
	\label{eq:equation2}
	\end{equation}	
	Но более интересным становится процесс интегрирования. Проинтегрируем функцию $f(x)$ на отрезке $[-10,10]$:
	
	

	\begin{multline}
	\int \limits_{-10}^{10} f(x)\,dx = \int \limits_{-10}^{-6} f(x)\,dx + \int \limits_{-6}^{-2} f(x)\,dx + \int \limits_{-2}^{2} f(x)\,dx + \int \limits_{2}^{6}\,dx + \int \limits_{6}^{10} f(x)\,dx =\\
	 =\int \limits_{-10}^{-6} x\,dx + \int \limits_{-6}^{-2} x^2\,dx + \int \limits_{-2}^{2} x^3\,dx + \int \limits_{2}^{6} x^4 \,dx + \int \limits_{6}^{10} x^5\,dx = \\
	= \eval{ \frac{x^2}{2} }_{-10}^{-6} + \eval{ \frac{x^3}{3} }_{-6}^{-2} + \eval{ \frac{x^4}{4} }_{-2}^{2} + \eval{ \frac{x^5}{5} }_{2}^{6} + \eval{ \frac{x^6}{6} }_{6}^{10}  = \\
 	= \frac{(-6)^2 - (-10)^2}{2} + \frac{(-2)^3 - (-6)^3}{3} + \frac{2^4 - (-2)^4}{4} + \frac{6^5 - 2^5}{5} + \frac{10^6 - 6^6}{6}  = \\
	= - \frac{64}{2} + \frac{208}{3} + 0 + \frac{7744}{5} + \frac{9536344}{6} = \\
	= -32 + 69.33 + 0 + 1548.8 + 1907268.8 = 1908854.93
	\end{multline}
	\section{Затмения}
	\hspace{17pt}Затмение — астрономическая ситуация, при которой одно небесное тело заслоняет свет от другого небесного тела. Наиболее известны лунные и солнечные затмения. Также существуют такие явления, как прохождения планет (Меркурия и Венеры) по диску Солнца.\par
	Солнечное затмение — астрономическое явление, которое заключается в том, что Луна (спутник земли) закрывает (затмевает) полностью или частично Солнце от наблюдателя на Земле. Солнечное затмение возможно только в новолуние, когда сторона Луны, обращённая к Земле, не освещена, и сама Луна не видна. \par
	Если Луна закрывает лишь часть солнечного диска, потому что их центры видны на некотором расстоянии друг от друга, то это частное затмение.\par
	Кольцеобразное затмение – это когда Луна полностью находит на Солнце, но оставляется еще и краешек Солнца, солнечные лучи пробиваются за Луну. Это происходит тогда, когда Луна меньше средней видимой величины, а это происходит, естественно, тогда, когда Луна дальше от Земли.
	\begin{figure}[p]
	\centering
	\begin{subfigure}{0.3\textwidth}
			\centering
			\includegraphics[scale = 2]{img/img17}
			\caption{Солнечное затмение}
	\end{subfigure}
	\hspace{0.1\textwidth}
	\begin{subfigure}{0.3\textwidth}
			\centering
			\includegraphics[width = 1.05\textwidth]{img/img16}
			\caption{Частичное затмение}		
	\end{subfigure} \\
	\begin{subfigure}{0.3\textwidth}
		\centering
		\includegraphics[width = \textwidth]{img/img15}
		\caption{Кольцеобразное затмение}
	\end{subfigure}
	\caption{Фотографии затмений}
	\end{figure}
\end{document}
