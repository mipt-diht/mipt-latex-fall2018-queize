\documentclass[11pt,a4paper]{article}
\usepackage[utf8]{inputenc}
\usepackage[russian]{babel}
\usepackage{gensymb}
\usepackage{wrapfig}
\usepackage{amsmath}
\usepackage{amssymb}
\usepackage{tabularx}
\usepackage{mathtools}
\usepackage{graphicx}
\usepackage{subcaption}
\usepackage[left=2cm,right=2cm,top=2cm,bottom=2cm]{geometry}
\usepackage[
	labelfont = bf,
	labelsep = period,	
	justification = centering,
	textfont = sl]{caption}
	
\newcolumntype{C}[0]{>{\centering\arraybackslash}X}
\renewcommand\tabularxcolumn[1]{m{#1}}

\begin{document}
	\section{Обзор планеты гиганта}
	\begin{figure}[h!]
		\centering
		\includegraphics[width = 0.4\textwidth]{img/saturn}
		\caption{Снимок планеты со станции <<Кассини>>}
		\label{pic:Saturn}
	\end{figure}	
	 \textbf{Сатурн} — шестая планета от Солнца и вторая по размерам планета в Солнечной системе после Юпитера. Сатурн, а также Юпитер, Уран и Нептун, классифицируются \textit{как газовые гиганты}. Сатурн назван в честь римского бога земледелия. На Рис.\,\ref{pic:Saturn} можно увидеть как выглядит фотография газового гиганта, сделанного при помощи автоматической станции <<Касини>> \par
	 В основном Сатурн состоит из водорода $\mathrm{H_2}$, с примесями гелия $\mathrm{He}$ и следами воды, метана $\mathrm{CH_4}$, аммиака $\mathrm{NH_3}$ и тяжёлых элементов. Внутренняя область представляет собой относительно небольшое ядро из железа $\mathrm{Fe}$, никеля $\mathrm{Ni}$ и льда, покрытое тонким слоем металлического водорода и газообразным внешним слоем. \par
	 Сатурн обладает заметной системой колец, состоящей главным образом из частичек льда, меньшего количества тяжёлых элементов и пыли. Вокруг планеты обращается 62 известных на данный момент спутника. \textbf{Титан} — самый крупный из них, а также второй по размерам спутник в Солнечной системе (после спутника Юпитера, Ганимеда), который превосходит по своим размерам Меркурий и обладает единственной среди спутников планет Солнечной системы плотной атмосферой. \par
	 \begin{figure}[h]
	 	\includegraphics[width = 0.6\textwidth]{img/img18}
	 	\centering
	 	\caption{Внутренее строение Сатурна}	
	 \end{figure}
	 В глубине атмосферы Сатурна растут давление и температура, а водород переходит в жидкое состояние, однако этот переход является постепенным. На глубине около 30 тыс. км водород становится металлическим (давление там достигает около 3 миллионов атмосфер). В центре планеты находится массивное ядро из твердых и тяжёлых материалов — силикатов, металлов и, предположительно, льда. Его масса составляет приблизительно от 9 до 22 масс Земли. Температура ядра достигает 11 700\,\degree C, а энергия, которую Сатурн излучает в космос, в 2,5 раза больше энергии, которую планета получает от Солнца.
	 \section{Спутники} 
	По состоянию на февраль 2010 г. известно 62 спутника Сатурна. Крупнейшие спутники --- Мимас, Энцелад, Тефия, Диона, Рея, Титан и Япет --- были открыты к 1789 году, однако и по сегодняшний день остаются основными объектами исследований. Распределение по массам соответствует распределению по диаметрам. Наибольшим эксцентриситетом орбиты обладает Титан, наименьшим — Диона и Тефия. Все спутники c известными параметрами находятся выше синхронной орбиты, что приводит к их постепенному удалению. \par 
	Самый крупный из спутников — \textit{Титан}. Также он является вторым по величине в Солнечной системе в целом, после спутника Юпитера Ганимеда. 
	\begin{figure}[h!]
		\begin{tabularx}{\textwidth}{|C|C|C|C|C|}
			\hline
			Название спутника & Большая полуось $a$, тыс.км & Эксцентриситет, $e$ & Период обращения $T$, земных суток & Наклон орбиты $i$, \degree \\
			\hline
			Мимас & 185,539 & 0,0196 & 0,942 & 1,574 \\
			\hline
			Энцелад & 237 948 & 0,0047 & 1,370 & 0,019 \\
			\hline
			Тефия & 294 672 & 0,0001 & 1,887802 & 1,12 \\
			\hline
			Диона & 377,4  & 0,0022 & 2,77 & 0,019 \\
			\hline
		\end{tabularx}
	\end{figure}
	\par \textsc{\LARGE Пока только 2 раздела. Скоро будет сделано два остальных!!!}
\end{document}
