\documentclass[12pt]{article}

\usepackage[russian]{babel}
\usepackage[utf8]{inputenc}
\title{Домашняя Работа №1}
\author{Николай Малышев}
\date{}

\begin{document}
	\maketitle
	\begin{flushright}
		\itshape{ Audi multa,\\loquere pauca}
	\end{flushright}
	\vspace{20pt}\par
	Это мой первый документ в компьютерной верстке \LaTeX 
	\begin{center}
		\sffamily\huge{<<УРА!!!>>}
	\end{center}\par
	А теперь формулы. {\scshape Формула}~--- краткое и точное словесное выражение, определение или же ряд  математических величин, выражений условными знаками.
	
	\vspace{15pt}
	\hspace{14pt} {\bfseries \Large Термодинамика} 
	
	Уравнение Менделеева--Клайперона~--- уравнение состояния идеального газа, имеющее вид $pV = \nu RT$, где $p$~--- это давление, $V$~--- это объем, занимаемый газом, $T$~--- температура газа, $\nu$~--- количество вещества, а $R$~--- универсальная газовая постоянная
	
	\vspace{15pt}
	\hspace{14pt} {\bfseries \Large Геометрия \hfill Планеметрия} 
	 
	Для {\slshape плоского} треугольника со сторонами $a$, $b$, $c$ и углом $\alpha$, лежащим против стороны $a$, справедливо соотношение
	$$ a^2 = b^2 + c^2 - 2bc \cos\alpha, $$ 
	из которого можно выразить косинус угла треугольника:
	$$
	\cos\alpha = {\frac{b^2 + c^2 - a^2}{2bc}}
	$$ \par
	Пусть $p$~--- полупериметр треугольника, тогда путем несложных преорбазований можно получить, что
	$$
	\tg\frac{\alpha}{2} = \sqrt{\frac{(p - b)(p - c)}{p(p - a)}}
	$$
	\vspace{1cm}
	\begin{flushleft}
		На сегодня, пожалуй, хватит\ldots Удачи!
	\end{flushleft}
\end{document}